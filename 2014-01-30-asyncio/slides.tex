\documentclass[20pt]{beamer}
\usepackage{fontspec}
\usepackage{amsfonts}
\usepackage{tangocolors}
\usepackage{listings}
\usepackage{hyperref}
\usepackage{multicol}

\usepackage{tikz}
\usetikzlibrary{shapes}

% thanks http://en.wikibooks.org/wiki/LaTeX/Packages/Listings
\newcommand\normallistingstyle{\lstset{ %
  language=Python,                % the language of the code
  basicstyle=\footnotesize,
                                   % the size of the fonts that are used for the code
  %numbers=left,                   % where to put the line-numbers
  %numberstyle=\tiny\color{gray},  % the style that is used for the line-numbers
  %stepnumber=1,                   % the step between two line-numbers. If it's 1, each line 
                                  % will be numbered
  %numbersep=5pt,                  % how far the line-numbers are from the code
  backgroundcolor=\color{white},      % choose the background color. You must add \usepackage{color}
  showspaces=false,               % show spaces adding particular underscores
  showstringspaces=false,         % underline spaces within strings
  showtabs=false,                 % show tabs within strings adding particular underscores
  %frame=single,                   % adds a frame around the code
  rulecolor=\color{black},        % if not set, the frame-color may be changed on line-breaks within not-black text
  tabsize=2,                      % sets default tabsize to 2 spaces
  %captionpos=b,                   % sets the caption-position to bottom
  breaklines=true,                % sets automatic line breaking
  breakatwhitespace=false,        % sets if automatic breaks should only happen at whitespace
  title=\ ,
  emph={[2]__import__,range,input,raw_input,NameError,dir,dict},
  keywordstyle=\color{ta3chocolate},          % keyword style
  commentstyle=\color{ta3orange},       % comment style
  stringstyle=\color{ta3orange},         % string literal style
  emphstyle={[2]\color{ta2orange}},
  escapeinside=\$\$,            % if you want to add LaTeX within your code
  morekeywords={*,...}               % if you want to add more keywords to the set
  aboveskip=0pt,
  belowskip=0pt,
}}
\newcommand\biglistingstyle{\lstset{ %
    basicstyle=\small,
}}
\newcommand\altlistingstyle{\lstset{ %
    backgroundcolor=\color{black},
    basicstyle=\small\color{white},
    keywordstyle=\color{tachocolate},
    commentstyle=\color{tachameleon},
    stringstyle=\color{tachameleon},
    emphstyle={[2]\color{tabutter}},
    emphstyle={[3]\color{taorange}},
}}
\normallistingstyle
\newcommand\topshade{
    \begin{tikzpicture}[remember picture,overlay]
    \fill[black] (current page.west) rectangle +(100cm, -100cm);
    \end{tikzpicture}
    \vspace{-50pt}
    \biglistingstyle
}
\newcommand\halfshadenopause{
    \altlistingstyle\bigskip\bigskip\bigskip
}
\newcommand\halfshade{
    \pause\halfshadenopause
}
\newcommand\bottomshade{
    \normallistingstyle
}
\newcommand\sk{\par\bigskip\bigskip\par}
\newcommand\wh[1]{\only<#1>{\color{white}}}
\newcommand\tx[2]{\alt<#1>{\textcolor{ta3gray}}{\textcolor{ta3gray}}{\uncover<#1->{#2}}}
\newcommand\rd[2]{\alt<#1>{\textcolor{taskyblue}}{\textcolor{ta3gray}}{#2}}
\renewcommand\emph[1]{\textcolor{taskyblue}{#1}}

\begin{document}
\fontspec[Numbers=Lining]{Fertigo Pro}
\color{ta3gray}

\begin{center}
\title{eval()}
\author{Petr Viktorin}
\date{\today}

\frame{\color{ta3gray}
    \sk
    \textcolor{ta2gray}{import}
    \textcolor{taskyblue}{asyncio}
    \sk\sk
    \textcolor{ta2gray}{Petr Viktorin}\\[-0.25cm]
    \textcolor{ta2gray}{\tiny encukou@gmail.com}
    \sk
    \textcolor{ta2gray}{\tiny Pyvo, 2014-01-30}
}

\frame{
    The \emph{problem}\only<3>{s}

    \bigskip

    Do something useful

    while waiting

    \pause (for IO)

    \bigskip
    \pause

    Do several things

    at the same time
}

\frame{
    Multiple processes

    \bigskip

    \small
    import \emph{subprocess}
}

\frame{
    Threads

    \bigskip

    \small
    import \emph{threading}
}

\frame{
    Greenlets

    \bigskip

    Event loop

    Cooperative multitacking

    Monkeypatching

    \bigskip

    \small
    pip install \emph{eventlet} \emph{gevent}
}

\frame{
    Callbacks

    \bigskip

    Explicit cooperative multitasking

    \bigskip

    \small
    pip install \emph{twisted}
}

\frame{
    Help! Which do I use?
}

\frame{
    Compare: web frameworks

    \bigskip

    \pause
    \emph{Flask} app

    \pause
    \emph{Pyramid} debug toolbar

    \pause
    \emph{Cherrypy} logging

    \bigskip

    \pause

    PEP 333 - Python \emph{W}eb \emph{S}erver \emph{G}ateway \emph{I}nterface
}

\frame{
    Let's do the same for async!

    \bigskip\bigskip\bigskip

    \pause
    import \emph{asyncio}

    \pause
    Python 3.4+

    \bigskip\bigskip\bigskip

    \pause
    pip install \emph{tulip}

    \pause
    Python 3.3+
}

\frame{
    \emph{asyncio}

    \bigskip\bigskip

    Explicit cooperative multitasking

    \pause
    \bigskip
    like Twisted
}

\frame{
    Global \emph{event loop}

    \bigskip

    \pause
    Event loop policy
}

\frame{
    Callbacks

    \bigskip

    \pause
    \texttt{call\_soon(callback, \*args)}

    \pause
    \texttt{call\_later(delay, cb, *args)}

    \pause
    \bigskip
    \tiny
    What is time?
}

\begin{frame}[fragile]
    Callback style

    \tiny
    \begin{verbatim}
    import asyncio

    def print_and_repeat(loop):
        print('Hello World')
        loop.call_later(2, print_and_repeat, loop)

    loop = asyncio.get_event_loop()
    loop.call_soon(print_and_repeat, loop)
    loop.run_forever()
    \end{verbatim}
\end{frame}

\frame{
    Futures

    \bigskip

    \pause
    Placeholders for a~result

    \bigskip

    \pause
    \texttt{future.result()}

    \pause
    \texttt{future.set\_result()}

    \pause
    \texttt{future.set\_exception()}

    \pause
    \texttt{future.add\_done\_callback()}

    \bigskip
    \pause
    \texttt{concurrent.futures}
}

\frame{
    Locks, Semaphores, Queues

    \bigskip\bigskip

    \pause
    \includegraphics[width=0.8\textwidth]{standards.png}

    \bigskip
    \tiny
    \url{http://xkcd.com/927/}
}

\begin{frame}[fragile]
    \emph{Couroutines} \& Tasks

    \tiny
    \begin{verbatim}
    import asyncio

    @asyncio.coroutine
    def greet_every_two_seconds():
        while True:
            print('Hello World')
            yield from asyncio.sleep(1)
            print('How are you?')
            yield from asyncio.sleep(1)

    loop = asyncio.get_event_loop()
    loop.run_until_complete(greet_every_two_seconds())
    \end{verbatim}

    %\texttt{asyncio.async(arg)}
\end{frame}

\frame{
    Transports \& Protocols

    \bigskip

    Transport: TCP, UDP, SSL

    Protocol: HTTP, FTP, IRC
}

\begin{frame}[fragile]
    Echo client

    \fontsize{8pt}{8pt}\selectfont
    \begin{verbatim}
    import asyncio

    class EchoClient(asyncio.Protocol):
        message = 'This is the message. It will be echoed.'

        def connection_made(self, transport):
            transport.write(self.message.encode())
            print('data sent: {}'.format(self.message))

        def data_received(self, data):
            print('data received: {}'.format(data.decode()))

        def connection_lost(self, exc):
            print('server closed the connection')
            asyncio.get_event_loop().stop()

    loop = asyncio.get_event_loop()
    coro = loop.create_connection(EchoClient, '127.0.0.1', 8888)
    loop.run_until_complete(coro)
    loop.run_forever()
    loop.close()
    \end{verbatim}
\end{frame}

\begin{frame}[fragile]
    Echo server

    \fontsize{8pt}{8pt}\selectfont
    \begin{verbatim}
    import asyncio

    class EchoServer(asyncio.Protocol):
        def connection_made(self, transport):
            peername = transport.get_extra_info('peername')
            print('connection from {}'.format(peername))
            self.transport = transport

        def data_received(self, data):
            print('data received: {}'.format(data.decode()))
            self.transport.write(data)

            # close the socket
            self.transport.close()

    loop = asyncio.get_event_loop()
    coro = loop.create_server(EchoServer, '127.0.0.1', 8888)
    server = loop.run_until_complete(coro)
    print('serving on {}'.format(server.sockets[0].getsockname()))

    try:
        loop.run_forever()
    except KeyboardInterrupt:
        print("exit")
    finally:
        server.close()
        loop.close()
    \end{verbatim}
\end{frame}

\frame{

    \bigskip\bigskip
    \bigskip\bigskip
    \bigskip\bigskip

    {\huge ?}

    \bigskip\bigskip
    \bigskip

    {\tiny
    Petr Viktorin\\[10pt]%
    \href{http://twitter.com/encukou}{@\rd{1}{encukou.cz}}\\%
    \href{http://twitter.com/encukou}{@\rd{1}{encukou}}\\%
    \href{mailto:encukou@gmail.com}{\rd{1}{encukou}@gmail.com}\\%
    \href{http://github.com/encukou}{github.com/\rd{1}{encukou}}\\%
    \sk
    \tx{1}{Slides available under the\\ Creative Commons Attribution-ShareAlike 3.0 \url{http://creativecommons.org/licenses/by-sa/3.0/}}\\
    }

}

\frame{
    \small
    \rd{1}{Sources} \& links\\[0.25cm]
    \bigskip\bigskip
    \tiny
    \tx{1}{\href{http://www.python.org/dev/peps/pep-3156/}{PEP 3156 - Asynchronous IO Support Rebooted: the "asyncio" Module}}\\[0.25cm]
    \tx{1}{\href{http://www.python.org/dev/peps/pep-3148/}{PEP 3148 - futures - execute computations asynchronously}}\\[0.25cm]
    \tx{1}{\href{http://www.python.org/dev/peps/pep-3153/}{PEP 3153 (Superseded) - Asynchronous IO support}}\\[0.25cm]
    \tx{1}{\url{http://docs.python.org/3.4/library/asyncio.html}}\\[0.25cm]
    \tx{1}{\href{http://www.youtube.com/watch?v=1coLC-MUCJc}{Youtube: Guido van Rossum - Tulip: Async I/O for Python 3}}\\[0.25cm]
    %\bigskip
    %\tx{1}{\url{}}\\[0.25cm]
    %\tx{1}{\url{}}\\[0.25cm]
    %\tx{1}{\url{}}\\[0.25cm]
    %\tx{1}{\url{}}\\[0.25cm]
   
}

\end{center}
\end{document}

