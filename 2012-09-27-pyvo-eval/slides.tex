\documentclass[20pt]{beamer}
\usepackage{fontspec}
\usepackage{amsfonts}
\usepackage{tangocolors}
\usepackage{listings}
\usepackage{multicol}

\usepackage{tikz}
\usetikzlibrary{shapes}

% thanks http://en.wikibooks.org/wiki/LaTeX/Packages/Listings
\newcommand\normallistingstyle{\lstset{ %
  language=Python,                % the language of the code
  basicstyle=\footnotesize,
                                   % the size of the fonts that are used for the code
  %numbers=left,                   % where to put the line-numbers
  %numberstyle=\tiny\color{gray},  % the style that is used for the line-numbers
  %stepnumber=1,                   % the step between two line-numbers. If it's 1, each line 
                                  % will be numbered
  %numbersep=5pt,                  % how far the line-numbers are from the code
  backgroundcolor=\color{white},      % choose the background color. You must add \usepackage{color}
  showspaces=false,               % show spaces adding particular underscores
  showstringspaces=false,         % underline spaces within strings
  showtabs=false,                 % show tabs within strings adding particular underscores
  %frame=single,                   % adds a frame around the code
  rulecolor=\color{black},        % if not set, the frame-color may be changed on line-breaks within not-black text
  tabsize=2,                      % sets default tabsize to 2 spaces
  %captionpos=b,                   % sets the caption-position to bottom
  breaklines=true,                % sets automatic line breaking
  breakatwhitespace=false,        % sets if automatic breaks should only happen at whitespace
  title=\ ,
  emph={[2]__import__,range,input,raw_input,eval,NameError},
  keywordstyle=\color{ta3chocolate},          % keyword style
  commentstyle=\color{ta3chameleon},       % comment style
  stringstyle=\color{ta3chameleon},         % string literal style
  emphstyle={[2]\color{ta3butter}},         % string literal style
  escapeinside=\$\$,            % if you want to add LaTeX within your code
  morekeywords={*,...}               % if you want to add more keywords to the set
  aboveskip=0pt,
  belowskip=0pt,
}}
\newcommand\biglistingstyle{\lstset{ %
    basicstyle=\small,
}}
\newcommand\altlistingstyle{\lstset{ %
    backgroundcolor=\color{black},
    basicstyle=\small\color{white},
    keywordstyle=\color{tachocolate},
    commentstyle=\color{tachameleon},
    stringstyle=\color{tachameleon},
    emphstyle={[2]\color{tabutter}},
}}
\normallistingstyle
\newcommand\topshade{
    \begin{tikzpicture}[remember picture,overlay]
    \fill[black] (current page.west) rectangle +(100cm, -100cm);
    \end{tikzpicture}
    \vspace{-50pt}
    \biglistingstyle
}
\newcommand\halfshadenopause{
    \altlistingstyle\bigskip\bigskip\bigskip
}
\newcommand\halfshade{
    \pause\halfshadenopause
}
\newcommand\bottomshade{
    \normallistingstyle
}
\newcommand\sk{\par\bigskip\bigskip\par}
\newcommand\wh[1]{\only<#1>{\color{white}}}
\newcommand\tx[2]{\alt<#1>{\textcolor{ta3gray}}{\textcolor{ta3gray}}{\uncover<#1->{#2}}}
\newcommand\rd[2]{\alt<#1>{\textcolor{ta2chameleon}}{\textcolor{ta2chameleon}}{\uncover<#1->{#2}}}

\begin{document}
\fontspec[Numbers=Lining]{Fertigo Pro}
\color{ta3gray}

\begin{center}
\title{eval()}
\author{Petr Viktorin}
\date{\today}

\frame{\color{ta3gray}
    \sk
    \textcolor{ta2gray}{Zákeřný}
    \textcolor{tachameleon}{eval()}
    \sk\sk
    \textcolor{ta2gray}{Petr Viktorin}\\[-0.25cm]
    \textcolor{ta2gray}{\tiny @encukou}
    \sk
    \textcolor{ta2gray}{\tiny nebezpečné PyVo 2012-09-27}
}

\frame{\color{ta3gray}
    {\fontspec{DejaVu Sans Mono}\tiny \$ pydoc eval \hfill}

    eval(\textit{source}[, \textit{globals}[, \textit{locals}]])
    \sk
    \rd{1}{Evaluate the source} in the context of globals and locals.
    \sk
    \small{
        The source may be a \rd{1}{string} representing a Python expression
        or a code object as returned by compile().
    }
}

\begin{frame}[fragile]\color{ta3gray}\tiny
\lstinputlisting[language=Python]{snippet/funceval.py}
\end{frame}

\begin{frame}[fragile]\color{ta3gray}\tiny
\begin{multicols}{2}
\begin{verbatim}
Zadej funkci: x * 5
x = 0, y = 0
x = 1, y = 5
x = 2, y = 10
x = 3, y = 15
x = 4, y = 20
\end{verbatim}
\pause
\columnbreak
\begin{verbatim}
Zadej funkci: math.sin(x) 
x = 0, y = 0.0
x = 1, y = 0.841470984808
x = 2, y = 0.909297426826
x = 3, y = 0.14112000806
x = 4, y = -0.756802495308
\end{verbatim}
\end{multicols}
\pause
\begin{verbatim}
Zadej funkci: x ** 999999
x = 0, y = 0
x = 1, y = 1
\end{verbatim}
\pause
\begin{verbatim}
Zadej funkci: open('/etc/passwd').read()
x = 0, y = root:x:0:0:root:/root:/bin/bash
bin:x:1:1:bin:/bin:/sbin/nologin
daemon:x:2:2:daemon:/sbin:/sbin/nologin
adm:x:3:4:adm:/var/adm:/sbin/nologin
\end{verbatim}
\pause
\begin{verbatim}
Zadej funkci: os.system('fire-mah-lazors --target=alderaan')
\end{verbatim}
\end{frame}

\frame{
Jak se \rd{1}{bránit}?
}

\frame{\color{ta3gray}
    {\fontspec{DejaVu Sans Mono}\tiny \$ pydoc eval \hfill}

    eval(\textit{source}[, \textit{globals}[, \textit{locals}]])
    \sk
    The \rd{1}{globals} must be a dictionary and \rd{1}{locals} can be any
    mapping, defaulting to the current globals and locals.
    \sk
    If only globals is given, locals defaults to it.
}

\begin{frame}[fragile]
\begin{lstlisting}
>>> import math
>>> eval("math.sin(3.141)")
0.0005926535550994539
$\pause$
>>> eval("math.sin(3.141)", {})
Traceback (most recent call last):
  ...
NameError: $name 'math' is not defined$

\end{lstlisting}
\end{frame}

\begin{frame}[fragile]
\begin{lstlisting}
>>> import math
>>> import os
>>> eval("math.sin(3.141)", dict(math=math))
0.0005926535550994539
$\pause$
>>> eval("os.system('echo pwned')", dict(math=math))
Traceback (most recent call last):
  ...
NameError: $name 'os' is not defined$

\end{lstlisting}
\end{frame}

\frame{Už je to \rd{1}{bezpečné}?}

\begin{frame}[fragile]
\begin{lstlisting}
>>> import math
>>> import os
>>> eval("dir()")
['__builtins__', '__doc__', '__name__', '__package__', 'math', 'os']
$\pause$
>>> eval("dir()", dict(math=math))
['__builtins__', 'math']
\end{lstlisting}
\end{frame}

\begin{frame}[fragile]
\begin{lstlisting}
>>> eval("dir(__builtins__)")
[..., 'abs', 'all', 'any', 'apply',
    'basestring', 'bin', 'bool',
    'buffer', 'bytearray', 'bytes',
    'callable', 'chr', 'cmp',
    ..., 'vars', 'xrange', 'zip']
\end{lstlisting}
\pause
\begin{lstlisting}
>>> eval('''open('/etc/passwd'
        ).read()''', {})
$'root:x:0:0:root:/root:/bin/bash{\textbackslash}n ...$
\end{lstlisting}
\end{frame}

\begin{frame}[fragile]
\begin{lstlisting}
>>> eval("open('/etc/passwd').read()", dict(__builtins__={}))
Traceback (most recent call last):
  ...
NameError: $name 'open' is not defined$
$\pause$
>>> eval("abs(-5)", dict(__builtins__={}, abs=abs))
5
\end{lstlisting}
\end{frame}

\frame{Už je to \rd{1}{bezpečné}!?}

\frame{eval()\\vyhodnotí\\\rd{1}{jakýkoli}\\výraz}

\begin{frame}[fragile]
Co \rd{1}{není} výraz?

\begin{lstlisting}
def f(x):
    return 3 + x
$\pause$
for x in range(10):
    print x
$\pause$
x = 3
$\pause$
import os
\end{lstlisting}
{\tiny a další příkazy: \url{http://docs.python.org/reference/index.html}}
\end{frame}

\frame{
Jak to \rd{1}{obejít}?
}

\begin{frame}[fragile]
\topshade
\begin{lstlisting}
def f(x):
    return 3 + x
\end{lstlisting}
\halfshade
\begin{lstlisting}
f = (lambda x: 3 + x)
$\color{black}!$
\end{lstlisting}
\bottomshade
\end{frame}

\begin{frame}[fragile]
\topshade
\begin{lstlisting}
x = 3 * 8
foobar(x)
\end{lstlisting}
\halfshade
\begin{lstlisting}
(lambda x=3 ** 8:
    foobar(x)
)()
\end{lstlisting}
\bottomshade
\end{frame}

\begin{frame}[fragile]
\topshade
\begin{lstlisting}
for x in range(10):
    print x
\end{lstlisting}
\halfshade
\begin{lstlisting}
[sys.stdout.write('%s\n' % x) for x in range(10)]
\end{lstlisting}
\bottomshade
\end{frame}

\begin{frame}[fragile]
\topshade
\begin{lstlisting}
class T(object):
    def __init__(self, p):
        self.p = p
\end{lstlisting}
\halfshade
\begin{lstlisting}
T = type("T", (object,),
  dict(
    __init__=lambda self, p:
      setattr(self, 'p', p)))
\end{lstlisting}
\bottomshade
\end{frame}

\begin{frame}[fragile]
\topshade
\begin{lstlisting}
import os
os.system("echo pwned")
\end{lstlisting}
\halfshade
\begin{lstlisting}
__import__('os').system(
    "echo pwned")
$\color{black}!$
\end{lstlisting}
\bottomshade
\end{frame}

\begin{frame}[fragile]
\topshade
\begin{lstlisting}
import os
os.system("echo pwned")
\end{lstlisting}
\halfshadenopause
\begin{lstlisting}
(lambda os=__import__('os'):
    os.system("echo pwned")
)()
\end{lstlisting}
\bottomshade
\end{frame}


\frame{
Téměř \rd{1}{jakýkoli} program se dá převést na jediný \rd{1}{výraz}

\bigskip\pause\bigskip

Ještě tak mít ten \rd{1}{\_\_import\_\_}...
}

\end{center}
\end{document}

