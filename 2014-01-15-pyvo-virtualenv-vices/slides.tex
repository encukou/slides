\documentclass[20pt]{beamer}
\usepackage{fontspec}
\usepackage{amsfonts}
\usepackage{tangocolors}
\usepackage{listings}
\usepackage{hyperref}
\usepackage{multicol}

\usepackage{tikz}
\usetikzlibrary{shapes}

% thanks http://en.wikibooks.org/wiki/LaTeX/Packages/Listings
\newcommand\normallistingstyle{\lstset{ %
  language=Python,                % the language of the code
  basicstyle=\footnotesize,
                                   % the size of the fonts that are used for the code
  %numbers=left,                   % where to put the line-numbers
  %numberstyle=\tiny\color{gray},  % the style that is used for the line-numbers
  %stepnumber=1,                   % the step between two line-numbers. If it's 1, each line 
                                  % will be numbered
  %numbersep=5pt,                  % how far the line-numbers are from the code
  backgroundcolor=\color{white},      % choose the background color. You must add \usepackage{color}
  showspaces=false,               % show spaces adding particular underscores
  showstringspaces=false,         % underline spaces within strings
  showtabs=false,                 % show tabs within strings adding particular underscores
  %frame=single,                   % adds a frame around the code
  rulecolor=\color{black},        % if not set, the frame-color may be changed on line-breaks within not-black text
  tabsize=2,                      % sets default tabsize to 2 spaces
  %captionpos=b,                   % sets the caption-position to bottom
  breaklines=true,                % sets automatic line breaking
  breakatwhitespace=false,        % sets if automatic breaks should only happen at whitespace
  title=\ ,
  emph={[2]__import__,range,input,raw_input,NameError,dir,dict},
  keywordstyle=\color{ta3chocolate},          % keyword style
  commentstyle=\color{ta3orange},       % comment style
  stringstyle=\color{ta3orange},         % string literal style
  emphstyle={[2]\color{ta2orange}},
  escapeinside=\$\$,            % if you want to add LaTeX within your code
  morekeywords={*,...}               % if you want to add more keywords to the set
  aboveskip=0pt,
  belowskip=0pt,
}}
\newcommand\biglistingstyle{\lstset{ %
    basicstyle=\small,
}}
\newcommand\altlistingstyle{\lstset{ %
    backgroundcolor=\color{black},
    basicstyle=\small\color{white},
    keywordstyle=\color{tachocolate},
    commentstyle=\color{tachameleon},
    stringstyle=\color{tachameleon},
    emphstyle={[2]\color{tabutter}},
    emphstyle={[3]\color{taorange}},
}}
\normallistingstyle
\newcommand\topshade{
    \begin{tikzpicture}[remember picture,overlay]
    \fill[black] (current page.west) rectangle +(100cm, -100cm);
    \end{tikzpicture}
    \vspace{-50pt}
    \biglistingstyle
}
\newcommand\halfshadenopause{
    \altlistingstyle\bigskip\bigskip\bigskip
}
\newcommand\halfshade{
    \pause\halfshadenopause
}
\newcommand\bottomshade{
    \normallistingstyle
}
\newcommand\sk{\par\bigskip\bigskip\par}
\newcommand\wh[1]{\only<#1>{\color{white}}}
\newcommand\tx[2]{\alt<#1>{\textcolor{ta3gray}}{\textcolor{ta3gray}}{\uncover<#1->{#2}}}
\newcommand\rd[2]{\alt<#1>{\textcolor{ta2plum}}{\textcolor{ta3gray}}{#2}}
\renewcommand\emph[1]{\textcolor{ta2plum}{#1}}

\begin{document}
\fontspec[Numbers=Lining]{Fertigo Pro}
\color{ta3gray}

\begin{center}
\title{eval()}
\author{Petr Viktorin}
\date{\today}

\frame{\color{ta3gray}
    \sk
    \textcolor{ta2gray}{Proč}
    \textcolor{taplum}{ne}\textcolor{ta2gray}{používat}
    \textcolor{ta2gray}{virtualenv}
    \sk\sk
    \textcolor{ta2gray}{Petr Viktorin}\\[-0.25cm]
    \textcolor{ta2gray}{\tiny encukou.cz}\\[-0.5cm]
    \textcolor{ta2gray}{\tiny encukou@gmail.com}
    \sk
    \textcolor{ta2gray}{\tiny Pražské Pyvo, 2014-01-15}
}

\frame{
    virtualenv je super!

    \bigskip

    \begin{tikzpicture}
        \pgfresetboundingbox
        \node[use as bounding box, color=white,text width=10cm, text centered] (a) {Psát Python bez \makebox{virtualenvu} je blbost};
        \draw[color=white,draw opacity=0] (node cs:name=a,anchor=north east) -- (node cs:name=a,anchor=south west);
        \only<2-3>{
            \node[text width=10cm, text centered] (a) {Psát Python bez \makebox{virtualenvu} je blbost!};
            \only<3>{
                \draw[line width=2pt, color=tascarletred] (node cs:name=a,anchor=north east) -- (node cs:name=a,anchor=south west);
            }
        }
    \end{tikzpicture}

    \pause

    \uncover<1,4>{ale ne na všechno}
}

\frame{
    \alt<1>{
        Instalace Pythonu\vphantom{Ay}
    }{
        S virtualenvem\vphantom{Ay}
    }

    \bigskip\bigskip\bigskip\bigskip

    \tiny
    \begin{tikzpicture}[x=0.25cm,y=0.5cm]
        \useasboundingbox (0, -2) -- (-28, 4) ++(0, 1em);

        \draw[ta2gray] (  2, -2) node[anchor=base west] {OS} rectangle +(-38, 1ex);
        \draw[ta2gray] (  2, -1) node[anchor=base west] {Python} rectangle +(-24, 1ex);
        \draw[ta2gray] (  2, 0) node[anchor=base west] {stdlib} rectangle +(-35, 1ex);
        \uncover<2>{
            \foreach\xsh in {-0.5cm,3cm} {
            \begin{scope}[xshift=\xsh]
                \draw[ta2plum] ( -7, 1.75) rectangle (-20, 4) ++(-0.5ex, 1ex) node[anchor=base west] {virtualenv};

                \draw[ta2gray] (-14, 2) node[anchor=base west] {Knihovny} rectangle +(-5, 1ex);
                \draw[ta2gray] (-15, 3) node[anchor=base west] {Program} rectangle +(-4, 1ex);

                \draw[<-] (-18, 2) +(0, 1ex) -- (-18, 3);
                \draw[<-] (-18, 0) +(0, 1ex) -- (-18, 2);
            \end{scope}
            }
        }
        \begin{scope}[xshift=-4cm]
            \draw[ta2gray] (-13, 1) node[anchor=base west] {Knihovny} rectangle +(-6, 1ex);
            \draw[ta2gray] (-15, 3) node[anchor=base west] {Program} rectangle +(-4, 1ex);
            \draw[<-] (-18, 1) +(0, 1ex) -- (-18, 3);
            \draw[<-] (-16, 0) +(0, 1ex) -- (-16, 1);
            \draw[<-] (-18, -1) +(0, 1ex) -- (-18, 1);
        \end{scope}

        \draw[ta2gray] ( -31, -1) node[anchor=base west] {C knihovny} rectangle +(-5, 1ex);

        \draw[<-] (-34, -2) +(0, 1ex) -- (-34, -1);
        \draw[<-] (-20, -2) +(0, 1ex) -- (-20, -1);

        \draw[<-] (-20, -1) +(0, 1ex) -- (-20, 0);
        \draw[<-] (-32, -1) +(0, 1ex) -- (-32, 0);
    \end{tikzpicture}
}

\begin{frame}[fragile]
    virtualenv vs. instalace pro \textcolor{taplum}{uživatele}

    \bigskip

    {\tiny
        \begin{verbatim}
        $ ~/virualenvs/md/bin/pip install markdown2
        $ ~/virualenvs/md/bin/markdown_py blogpost.md
        \end{verbatim}
    }

    vs

    {\tiny
        \begin{verbatim}
        $ pip install --user markdown2
        $ markdown_py blogpost.md
        \end{verbatim}
    }
\end{frame}

\begin{frame}[fragile]
    virtualenv vs. \textcolor{taplum}{systémové} balíky

    \tiny
    \begin{verbatim}
    $ sudo yum install python-markdown2
    $ markdown_py blogpost.md
    \end{verbatim}
\end{frame}

\begin{frame}[fragile]
    virtualenv vs. samotný \textcolor{taplum}{systém}

    \bigskip\bigskip

    \tiny
    \begin{verbatim}
    $ head -n4 /usr/bin/yum
    #!/usr/bin/python
    import sys
    try:
        import yum
    $
    \end{verbatim}

    \bigskip
\end{frame}

\frame{
    virtualenv vs. \textcolor{taplum}{virtualizace}~a~kontejnery
}

\newcommand\dockerbox[1]{
    \begin{scope}[xshift=#1]
    \draw[ta2plum] ( -8, 0.75) rectangle (-20, 4) ++(-0.5ex, 1ex) node[anchor=base west] {overlay filesystem};

    \draw[ta2gray] (-14, 1) node[anchor=base west] {System} rectangle +(-5, 1ex);
    \draw[ta2gray] (-15, 2) node[anchor=base west] {Python} rectangle +(-4, 1ex);
    \draw[ta2gray] (-16, 3) node[anchor=base west] {App} rectangle +(-3, 1ex);

    \draw[<-] (-18, 1) +(0, 1ex) -- (-18, 2);
    \draw[<-] (-18, 2) +(0, 1ex) -- (-18, 3);
    \end{scope}
}

\frame{
    docker.io

    \bigskip

    \tiny
    \begin{tikzpicture}[x=0.25cm,y=0.5cm]
    \path[use as bounding box] (0, 0) -- (-28, 4) ++(0, 1em);

    \draw[ta2gray] (  0, 0) node[anchor=base west] {Kernel} rectangle +(-28, 1ex);
    \dockerbox{0}
    \pause
    \dockerbox{3.5cm}
    \dockerbox{-3.5cm}
    \pause
    \draw[ta2plum,line width=5pt] (-7, 0.5) -- +(0, 4.5);
    \draw[ta2plum,line width=5pt] (-21, 0.5) -- +(0, 4.5);
    \end{tikzpicture}
}

\frame{

    \bigskip\bigskip
    \bigskip\bigskip
    \bigskip\bigskip

    {\huge ?}

    \bigskip\bigskip
    \bigskip

    {\tiny
    Petr Viktorin\\[10pt]%
    \href{http://encukou.cz}{@\rd{1}{encukou}.cz}\\%
    \href{mailto:encukou@gmail.com}{\rd{1}{encukou}@gmail.com}\\%
    \href{http://twitter.com/encukou}{@\rd{1}{encukou}}\\%
    \href{http://github.com/encukou}{github.com/\rd{1}{encukou}}\\%
    \sk
    \tx{1}{Licence: \\ Creative Commons Attribution-ShareAlike 3.0 \url{http://creativecommons.org/licenses/by-sa/3.0/}}\\
    }

}

% \frame{
%     \small
%     \rd{1}{Zdroje} \& odkazy\\[0.25cm]
%     \bigskip\bigskip
%     \tiny
%     %\tx{1}{\url{}}\\[0.25cm]
%     %\tx{1}{\url{}}\\[0.25cm]
%     %\bigskip
%     %\tx{1}{\url{}}\\[0.25cm]
%     %\tx{1}{\url{}}\\[0.25cm]
%     %\tx{1}{\url{}}\\[0.25cm]
% }

\end{center}
\end{document}

