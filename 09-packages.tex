\documentclass{pyslides}

\usepackage{multicol}

\title{Packages}
\pyslidenumber{9}
\date{November 2010}

\newcommand\im[1]{\par\vspace{3pt}\hspace{0ex}\rlap{\tt #1}\hspace{3.5cm}}
\newcommand\imz[1]{\par\vspace{3pt}\hspace{0ex}\rlap{\tt #1}\hspace{1.5cm}}

%%%%%%%%%%%%%%%%%%%%%%%%%%%%%%%%%%%%%%%%%%%%%%%%%%%
\begin{document}

\begin{frame}\titlepage\end{frame}

\section{Creating}

\begin{frame}[fragile]{Creating Packages}
A \emph{package} is a collection of modules.

\bigskip

To create a package, put a file named \texttt{\magic{init}.py} into a directory.
\end{frame}

\begin{frame}[fragile]{Sound Package Example}
\begin{columns}[t]
\begin{column}{4cm}
\vspace{-0.25cm}
\begin{Verbatim}[fontsize=\small]
sound/
    __init__.py
    formats/
        __init__.py
        wav.py
        aiff.py
        au.py
    effects/
        __init__.py
        echo.py
        surround.py
        reverse.py
    filters/
        __init__.py
        equalizer.py
        vocoder.py
        karaoke.py
\end{Verbatim}
\bigskip
\end{column}
\begin{column}{6cm}
To use this package:
\input "|./highlight.py '=import sound.effects.echo' fontsize=!small no"

or

\input "|./highlight.py '=from sound.effects import echo' fontsize=!small no"
\end{column}
\end{columns}
\end{frame}

\begin{frame}[fragile]{\magic{init}.py}
\input "|./highlight.py '=from sound import effects' fontsize=!small no"

When you import a package, you are importing the coresponding \texttt{\magic{init}.py} module.

This module's variables are available as attributes, as with any other module.

\bigskip

Note that this does not give you the package's submodules (unless \texttt{\magic{init}.py} imports them).
\end{frame}

\section{Packaging}

\begin{frame}[fragile]{Distributing packages}
If you want to distribute your package to others, use the \texttt{setuptools} library
to make it installable and add some information about it.

\bigskip

Look at the library's documentation, or see how other libraries you use do it.
\end{frame}

\begin{frame}[fragile]{setup.py}
\input "|./highlight.py 'samples/09setup.py' fontsize=!tiny no"
\end{frame}

\section{Installing}

\begin{frame}[fragile]{Virtual environment}
When trying out Python packages, it can be useful to install them to a “virtual environment”:
\begin{itemize}
\item You don't have to be a system administrator (root)
\item If something goes wrong, just delete one folder
\end{itemize}
\end{frame}

\begin{frame}[fragile]{Installing \& Using virtualenv}
\begin{itemize}
\item Download \href{http://pypi.python.org/pypi/virtualenv}{pypi.python.org/pypi/virtualenv}
\item Run \texttt{virtualenv \emph{environment-name}}
\item Run \texttt{environment-name/bin/activate} to activate
\end{itemize}
Packages installed in a virtualenv will only be available in that virtualenv.
\end{frame}

\begin{frame}[fragile]{PIP}
\emph{pip installs packages.}

\bigskip

If you use virtualenv, it is installed by default.

Otherwise, download it from \href{http://pypi.python.org/pypi/pip}{pypi.python.org/pypi/pip}.

\bigskip

\begin{itemize}
\item \texttt{pip install \emph{package} } installs a package
\item \texttt{pip help } lists all options
\end{itemize}

\end{frame}

\end{document}
