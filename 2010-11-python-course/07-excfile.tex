\documentclass{pyslides}

\title{Exceptions \& File Handling}
\pyslidenumber{7}
\date{November 2010}

\newcommand\im[1]{\par\vspace{3pt}\hspace{0ex}\rlap{\tt #1}\hspace{3.5cm}}
\newcommand\imz[1]{\par\vspace{3pt}\hspace{0ex}\rlap{\tt #1}\hspace{1.5cm}}

%%%%%%%%%%%%%%%%%%%%%%%%%%%%%%%%%%%%%%%%%%%%%%%%%%%
\begin{document}

\begin{frame}\titlepage\end{frame}

\section{File handling}

\begin{frame}[fragile]{Opening files}
The open() function opens files. It can take two arguments:
\bigskip
\input "|./highlight.py '=open(filename, mode=`apos*r*apos`)' fontsize=!small no"
\bigskip
Mode can be:
\imz{r} Open for reading
\imz{w} Open for writing
\imz{a} Open for appending

\smallskip

To which you can add:
\imz{b} for binary files.
\imz{+} to allow both reading and writing.
\imz{U} to convert all newline styles to \verb}\n}.

\bigskip

For example:
\input "|./highlight.py '=open(`apos*out.dat*apos`, `apos*wb*apos`)' fontsize=!small no"

\end{frame}

\begin{frame}[fragile]{File objects}
Iterating over a file yields the lines in that file

\bigskip

Using a file in a \verb+with+ block automatically closes it at the end

\bigskip

Useful file methods:

\im{f.read()} Read the contents
\im{f.readline()} Read one line
\im{f.read(x)} Read x bytes
\im{f.write(s)} Write a string to the file
\im{f.close()} Close the file

\end{frame}

\begin{frame}[fragile]{Other file objects}
The \verb+sys+ module includes objects for standard input, output and error streams.

\bigskip

\input "|./highlight.py 'samples/07streams.py' fontsize=!small"
\end{frame}

\begin{frame}[fragile]{Other file-like objects}
Remember \verb+urllib.urlopen+? It doesn't quite return a file object, but whatever it does return acts like a file.
\end{frame}

\begin{frame}[fragile]{Parsing something?}
There's usually a library for your file format: INI, YAML, CSV, MAT, ...

Don't reinvent the wheel.
\end{frame}

\section{Exceptions}

\begin{frame}[fragile]{When something goes wrong}
When there is an error or just an unusual state, an \emph{exception} is raised.
\bigskip
\input "|./highlight.py 'samples/07exceptions.txt' fontsize=!small,lastline=4"
\end{frame}

\begin{frame}[fragile]{Handling exceptions}
Python allows us to handle these exceptions.
\bigskip
\input "|./highlight.py 'samples/07exceptions.py' fontsize=!small"
\end{frame}

\begin{frame}[fragile]{Raising exceptions}
Python also allows us to raise exceptions.
\bigskip
\input "|./highlight.py 'samples/07security.py' fontsize=!small"
\end{frame}

\begin{frame}[fragile]{Cleanup with finally}
A \texttt{finally} clause will execute every time.
\bigskip
\input "|./highlight.py 'samples/07closefile.py' fontsize=!small"
\end{frame}

\begin{frame}[fragile]{Context managers}
Some objects have a \texttt{try/finally} clause “built in”. These can used with the \verb}with} statement:
\bigskip
\input "|./highlight.py 'samples/07with.py' fontsize=!small"
\end{frame}

\begin{frame}[fragile]{Context managers II}
Context managers are used whenever there's a “begin/end” pair of functions; for example in multi-thread locking
\bigskip
\input "|./highlight.py 'samples/07with2.py' fontsize=!small"
\end{frame}

\begin{frame}[fragile]{What exception to use?}
There are many standard exceptions: \href{http://docs.python.org/library/exceptions.html}{http://docs.python.org/library/exceptions.html}.

Which one to use?

\bigskip

The rule of thumb:
\begin{itemize}
\item Handle the specific exception that's raised when you test
\item Raise something appropriate – or better yet, make your own exception
\end{itemize}
\end{frame}

\begin{frame}[fragile]{Subclassing Exception}
\input "|./highlight.py 'samples/08customexc.py' fontsize=!small"

\bigskip

SecurityException is now a subclass of IOError, which itself is a subclas of Exception.

This means \verb+except IOError+ or \verb+except Exception+ will handle it.
\end{frame}

\end{document}
