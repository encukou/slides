\documentclass{pyslides}

\usepackage{multicol}

\title{Some final slides}
\pyslidenumber{10}
\date{November 2010}

\newcommand\im[1]{\par\vspace{3pt}\hspace{0ex}\rlap{\tt #1}\hspace{3.5cm}}
\newcommand\imz[1]{\par\vspace{3pt}\hspace{0ex}\rlap{\tt #1}\hspace{1.5cm}}

%%%%%%%%%%%%%%%%%%%%%%%%%%%%%%%%%%%%%%%%%%%%%%%%%%%
\begin{document}

\begin{frame}\titlepage\end{frame}

\section{}

\begin{frame}[fragile]{What did we miss?}
There's always more to learn...

\bigskip

(Or: I forgot to tell you some things)

\bigskip

Anyway, here are Decorators and a few tips
\end{frame}

\section{Decorators}

\begin{frame}[fragile]{Class method}
\input "|./highlight.py 'samples/10decorators.py' fontsize=!small"
\end{frame}

\begin{frame}[fragile]{Static method}
\input "|./highlight.py 'samples/10decorators2.py' fontsize=!small"
\end{frame}

\begin{frame}[fragile]{How decorators work}
A decorator is a function that takes a function, and returns something to replace it.

The following are equivalent:

\input "|./highlight.py 'samples/10decorators-ex.py' fontsize=!small no"
\end{frame}

\begin{frame}[fragile]{Custom wrapping decorator}
\input "|./highlight.py 'samples/10decorators3.py' fontsize=!small"
\end{frame}

\begin{frame}[fragile]{Properties}
\input "|./highlight.py 'samples/10decorators2a.py' fontsize=!small"
\end{frame}

\section{Tips \& Tricks}

\begin{frame}[fragile]{List of strings}
\input "|./highlight.py 'samples/10listofstrings.txt' fontsize=!small"
\end{frame}

\begin{frame}[fragile]{Good practice}
\begin{itemize}
\item Name your variables consistently.
\item Don't use names of built-in objects.
\item Indent each block with four spaces.
\item Write good docstrings.
\item Use language features that apply. For example, iterate over anything iterable.
\item It's better to beg for forgiveness than to ask for permission.
\item Numeric zeroes and empty containers are false
\item Use the standard library!
\item Follow PEP8 to make your code look perfect.
\\\url{http://www.python.org/dev/peps/pep-0008/}
\end{itemize}

\end{frame}

\end{document}
