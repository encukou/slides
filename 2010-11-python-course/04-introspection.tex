\documentclass{pyslides}

\title{The Power of Introspection}
\pyslidenumber{4}
\date{November 2010}

%%%%%%%%%%%%%%%%%%%%%%%%%%%%%%%%%%%%%%%%%%%%%%%%%%%
\begin{document}

\begin{frame}\titlepage\end{frame}

\section{Apihelper}

\begin{frame}[fragile]{apihelper.py}
Download apihelper.py from the course page.
\bigskip
\input "|./highlight.py samples/04introspection.py fontsize=!tiny"
\end{frame}

\begin{frame}[fragile]{apihelper.py}
Testing apihelper.py
\bigskip
\input "|./highlight.py samples/04introspecttest.txt fontsize=!tiny no"
\end{frame}

\section{Calling functions}

\begin{frame}[fragile]{Using Optional and Named Arguments}
Let's look at the function declaration:
\input "|./highlight.py samples/04introspection.py lastline=1"
\bigskip
You can call this in many different ways:
\input "|./highlight.py samples/04calling.py lastline=7 no"
\end{frame}

\begin{frame}[fragile]{Using Sequences and Dictionaries}
\input "|./highlight.py samples/04introspection.py lastline=1"
\bigskip
If you have more arguments stored in a list or dict, use stars:
\input "|./highlight.py samples/04calling.py firstline=9,lastline=13 no"
\end{frame}

\begin{frame}[fragile]{Declaring Functions with argument lists}
The same can be used in function definitions:
\input "|./highlight.py samples/04calling.py firstline=15 no"
\end{frame}

\section{Built-in functions}

\begin{frame}[fragile]{Built-in functions}
Several Python functions are so useful, you don't have to import them from a module. We already know some:
\begin{itemize}
\item str
\item list
\item dict
\item tuple
\end{itemize}
Now we are going to cover:
\begin{itemize}
\item type
\item dir
\item callable
\item getattr
\end{itemize}
\end{frame}

\begin{frame}[fragile]{The type Function}
\input "|./highlight.py samples/04builtinfuncs.txt lastline=11 no"
\end{frame}

\begin{frame}[fragile]{The dir Function}
\input "|./highlight.py samples/04builtinfuncs.txt firstline=13,lastline=27 no"
\end{frame}

\begin{frame}[fragile]{The callable Function}
\input "|./highlight.py samples/04builtinfuncs.txt firstline=29,lastline=43 no"
\end{frame}

\begin{frame}[fragile]{The range Function}
\input "|./highlight.py samples/04builtinfuncs.txt fontsize=!small,firstline=45,lastline=50 no"
\end{frame}

\begin{frame}[fragile]{The getattr Function}
\input "|./highlight.py samples/04getattr.txt fontsize=!small,lastline=14 no"
\end{frame}

\begin{frame}[fragile]{All built-in functions}
For all built-in functions, look in the {\tt\magic{builtins}} module:
\input "|./highlight.py samples/04builtinfuncs.txt fontsize=!small,firstline=52 no"
\bigskip
Or use the \href{http://docs.python.org/library/functions.html}{Python manual}.
\end{frame}

\section{Filtering}

\begin{frame}[fragile]{Filtering list comprehensions}
We already covered list comprehensions, now we extend the syntax a bit:
\input "|./highlight.py samples/04filtering.txt fontsize=!small no"
\end{frame}

\begin{frame}[fragile]{Filtering out special methods}
Now let's look back at the filtering line in apihelper:
\input "|./highlight.py samples/04introspection.py fontsize=!scriptsize,firstline=8,lastline=8,firstnumber=8,gobble=4"
\bigskip
The usual naming conventions in Python are like this:
\begin{itemize}
\item No underscores for normal, “public” attributes: \verb+info+
\item One underscores for implementation details: \verb+_do_xyz+
\item Two underscores for hidden, “private” attributes: \verb+__parse+
\item Two underscores on each side for special attributes defined by Python itself: \verb+__init__+
\end{itemize}
In any case, attributes that begin with an underscore are likely to be uninteresting. That's why we filter them out.
\end{frame}

\section{Lambdas}

\begin{frame}[fragile]{Lambda Functions}
\input "|./highlight.py samples/04lambda.txt fontsize=!small,lastline=10 no"
\end{frame}

\begin{frame}[fragile]{Using Lambdas in Sorting}
\input "|./highlight.py samples/04lambda.txt fontsize=!small,firstline=12,lastline=12 no"
\input "|./highlight.py samples/04lambda.txt fontsize=!small,firstline=14 no"
\end{frame}

\section{Wrapping Up}

\begin{frame}[fragile]{apihelper.py}
\input "|./highlight.py samples/04introspection.py fontsize=!tiny"
\end{frame}

\begin{frame}[fragile]{Why use str() on the docstring}
\input "|./highlight.py samples/04wrapup.txt fontsize=!small,lastline=9 no"
\end{frame}

\begin{frame}[fragile]{Reinventing the wheel}
Actually, apihelper.py isn't that useful. For exapmle, it only lists methods:
other things might have docstings as well.

It turns out Python already has a tool that
does much of the same thing, but better: the pydoc module.
\bigskip
\input "|./highlight.py samples/04wrapup.txt fontsize=!scriptsize,firstline=11,lastline=20 no"
\end{frame}

\begin{frame}[fragile]{Built-in Help}
Pydoc's help function is built in: you don't actually need to import pydoc.

Also, you can get detailed help on some things that are not Python objects, like statements.
\bigskip
\input "|./highlight.py samples/04wrapup.txt fontsize=!scriptsize,firstline=22 no"
\end{frame}



\end{document}
