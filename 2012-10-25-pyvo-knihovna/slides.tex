\documentclass[20pt]{beamer}
\usepackage{fontspec}
\usepackage{amsfonts}
\usepackage{tangocolors}
\usepackage{listings}
\usepackage{hyperref}
\usepackage{multicol}

\usepackage{tikz}
\usetikzlibrary{shapes}

% thanks http://en.wikibooks.org/wiki/LaTeX/Packages/Listings
\newcommand\normallistingstyle{\lstset{ %
  language=Python,                % the language of the code
  basicstyle=\footnotesize,
                                   % the size of the fonts that are used for the code
  %numbers=left,                   % where to put the line-numbers
  %numberstyle=\tiny\color{gray},  % the style that is used for the line-numbers
  %stepnumber=1,                   % the step between two line-numbers. If it's 1, each line 
                                  % will be numbered
  %numbersep=5pt,                  % how far the line-numbers are from the code
  backgroundcolor=\color{white},      % choose the background color. You must add \usepackage{color}
  showspaces=false,               % show spaces adding particular underscores
  showstringspaces=false,         % underline spaces within strings
  showtabs=false,                 % show tabs within strings adding particular underscores
  %frame=single,                   % adds a frame around the code
  rulecolor=\color{black},        % if not set, the frame-color may be changed on line-breaks within not-black text
  tabsize=2,                      % sets default tabsize to 2 spaces
  %captionpos=b,                   % sets the caption-position to bottom
  breaklines=true,                % sets automatic line breaking
  breakatwhitespace=false,        % sets if automatic breaks should only happen at whitespace
  title=\ ,
  emph={[2]__import__,range,input,raw_input,NameError,dir,dict},
  keywordstyle=\color{ta3chocolate},          % keyword style
  commentstyle=\color{ta3orange},       % comment style
  stringstyle=\color{ta3orange},         % string literal style
  emphstyle={[2]\color{ta2orange}},
  escapeinside=\$\$,            % if you want to add LaTeX within your code
  morekeywords={*,...}               % if you want to add more keywords to the set
  aboveskip=0pt,
  belowskip=0pt,
}}
\newcommand\biglistingstyle{\lstset{ %
    basicstyle=\small,
}}
\newcommand\altlistingstyle{\lstset{ %
    backgroundcolor=\color{black},
    basicstyle=\small\color{white},
    keywordstyle=\color{tachocolate},
    commentstyle=\color{tachameleon},
    stringstyle=\color{tachameleon},
    emphstyle={[2]\color{tabutter}},
    emphstyle={[3]\color{taorange}},
}}
\normallistingstyle
\newcommand\topshade{
    \begin{tikzpicture}[remember picture,overlay]
    \fill[black] (current page.west) rectangle +(100cm, -100cm);
    \end{tikzpicture}
    \vspace{-50pt}
    \biglistingstyle
}
\newcommand\halfshadenopause{
    \altlistingstyle\bigskip\bigskip\bigskip
}
\newcommand\halfshade{
    \pause\halfshadenopause
}
\newcommand\bottomshade{
    \normallistingstyle
}
\newcommand\sk{\par\bigskip\bigskip\par}
\newcommand\wh[1]{\only<#1>{\color{white}}}
\newcommand\tx[2]{\alt<#1>{\textcolor{ta3gray}}{\textcolor{ta3gray}}{\uncover<#1->{#2}}}
\newcommand\rd[2]{\alt<#1>{\textcolor{ta2orange}}{\textcolor{ta3gray}}{#2}}
\renewcommand\emph[1]{\textcolor{ta2orange}{#1}}

\begin{document}
\fontspec[Numbers=Lining]{Fertigo Pro}
\color{ta3gray}

\begin{center}
\title{eval()}
\author{Petr Viktorin}
\date{\today}

\frame{\color{ta3gray}
    \sk
    \textcolor{ta2gray}{Jak}
    \textcolor{taorange}{zveřejnit}
    \textcolor{ta2gray}{knihovnu}
    \sk\sk
    \textcolor{ta2gray}{Petr Viktorin}\\[-0.25cm]
    \textcolor{ta2gray}{\tiny encukou@gmail.com}
    \sk
    \textcolor{ta2gray}{\tiny PyVo, 2012-10-25}
}

\frame{
    Licence

    \bigskip\bigskip
    \bigskip\bigskip

    BSD? \rd{1}{MIT}? GPL?

    \bigskip\bigskip
    \bigskip\bigskip
}

\frame{
    Git \& Github

    \pause

    nebo jinde?

}

\begin{frame}[fragile]
Struktura projektu
\begin{lstlisting}
LICENSE             $\pause$
README              $\pause$
CHANGELOG           $\pause$
$\rd{1}{setup.py}$  $\pause$
$\rd{1}{mojeknihovna}$/
    __init__$.py$   $\pause$
    tests/          $\pause$
doc/                $\pause$
bin/
\end{lstlisting}
\bigskip\bigskip
\end{frame}

\begin{frame}[fragile]
setup.py
\lstset{basicstyle=\tiny}
\begin{lstlisting}
from setuptools import setup, find_packages

setup(
    name='Moje knihovna',
    version='0.0.1',
    author='Já První',
    author_email='ja@example.com',
    scripts=['bin/necodelej'],
    url='http://github.com/ja/mojeknihovna/',
    license='MIT',
    description='Moje první knihovna.',
    install_requires=[
        "docopt >= 0.5.0",
    ],
    packages=find_packages(),
    long_description=open('README').read(),
    classifiers=[$\rd{1}{...}$],
)
\end{lstlisting}
\normallistingstyle
\bigskip\bigskip
\end{frame}

\frame{
    Trove Classifiers

    \bigskip\bigskip

    \small
    \url{http://pypi.python.org/pypi?:action=list_classifiers}

    \bigskip\bigskip
    \footnotesize

    Natural Language :: Czech

    \bigskip

    License :: OSI Approved :: MIT License

    \bigskip

    Programming Language :: Python :: 3

}

\frame{
    Nestačí?

    \bigskip\bigskip

    \url{http://packages.python.org/distribute/setuptools.html}
}

\begin{frame}[fragile]
Na PyPI s tím!

\bigskip\bigskip

\url{http://pypi.python.org/pypi}

\bigskip\bigskip

\begin{itemize}\itemsep 10pt
\item[]\hspace{-20pt}\$ python setup.py \rd{1}{register} \\
\item[]\hspace{-20pt}\$ python setup.py \rd{1}{sdist} \rd{1}{upload} \\
\end{itemize}
\end{frame}

\frame{
    A je to!

\bigskip\bigskip

    \rd{1}{pip \ install} \ mojeknihovna

}

\frame{
    Verzování

\bigskip\bigskip
\bigskip\bigskip

    {\huge \rd{3}{0}.\rd{2}{3}.\rd{1}{7}}

\bigskip\bigskip

}

\frame{
    \rd{1}{Dokumentace}

\bigskip\bigskip

    Sphinx

\bigskip

    readthedocs.org
}

\frame{
    \emph{Propagace}\\\pause

\bigskip\bigskip

Blogy \\\pause
Twittery \\\pause
Géplusy \\\pause
Xichtobichle \\\pause

\bigskip

Přednášky \\\pause
Workshopy \\\pause
...

}

\frame{

    \bigskip\bigskip
    \bigskip\bigskip
    \bigskip\bigskip

    {\huge ?}

    \bigskip\bigskip
    \bigskip

    {\tiny
    Petr Viktorin\\[10pt]%
    \href{http://twitter.com/encukou}{@\rd{1}{encukou}}\\%
    \href{mailto:encukou@gmail.com}{\rd{1}{encukou}@gmail.com}\\%
    \href{http://github.com/encukou}{github.com/\rd{1}{encukou}}\\%
    \sk
    \tx{1}{Licence slajdů \& videa: \\ Creative Commons Attribution-ShareAlike 3.0 \url{http://creativecommons.org/licenses/by-sa/3.0/}}\\
    }

}

\frame{
    \small
    \rd{1}{Zdroje} \& odkazy\\[0.25cm]
    \bigskip\bigskip
    \tiny
    \tx{1}{\url{http://as.ynchrono.us/2007/12/filesystem-structure-of-python-project_21.html}}\\[0.25cm]
    \tx{1}{\url{http://packages.python.org/distribute/setuptools.html}}\\[0.25cm]
    \tx{1}{\url{http://guide.python-distribute.org/creation.html}}\\[0.25cm]
    \bigskip
    \tx{1}{\url{http://pypi.python.org/pypi}}\\[0.25cm]
    \bigskip
    \tx{1}{\url{http://semver.org/}}\\[0.25cm]
    \bigskip
    \tx{1}{\url{http://sphinx.pocoo.org/}}\\[0.25cm]
    \tx{1}{\url{https://readthedocs.org/}}\\[0.25cm]
    %\tx{1}{\url{}}\\[0.25cm]
    %\tx{1}{\url{}}\\[0.25cm]
    %\tx{1}{\url{}}\\[0.25cm]
    %\tx{1}{\url{}}\\[0.25cm]
   
}

\end{center}
\end{document}

