\documentclass[20pt]{beamer}
\usepackage{fontspec}
\usepackage{amsfonts}
\usepackage{tangocolors}
\usepackage{listings}
\usepackage{hyperref}
\usepackage{multicol}

\usepackage{tikz}
\usetikzlibrary{shapes}

% thanks http://en.wikibooks.org/wiki/LaTeX/Packages/Listings
\newcommand\normallistingstyle{\lstset{ %
  language=Python,                % the language of the code
  basicstyle=\footnotesize,
                                   % the size of the fonts that are used for the code
  %numbers=left,                   % where to put the line-numbers
  %numberstyle=\tiny\color{gray},  % the style that is used for the line-numbers
  %stepnumber=1,                   % the step between two line-numbers. If it's 1, each line 
                                  % will be numbered
  %numbersep=5pt,                  % how far the line-numbers are from the code
  backgroundcolor=\color{white},      % choose the background color. You must add \usepackage{color}
  showspaces=false,               % show spaces adding particular underscores
  showstringspaces=false,         % underline spaces within strings
  showtabs=false,                 % show tabs within strings adding particular underscores
  %frame=single,                   % adds a frame around the code
  rulecolor=\color{black},        % if not set, the frame-color may be changed on line-breaks within not-black text
  tabsize=2,                      % sets default tabsize to 2 spaces
  %captionpos=b,                   % sets the caption-position to bottom
  breaklines=true,                % sets automatic line breaking
  breakatwhitespace=false,        % sets if automatic breaks should only happen at whitespace
  title=\ ,
  emph={[2]__import__,range,input,raw_input,NameError,dir,dict},
  keywordstyle=\color{ta3chocolate},          % keyword style
  commentstyle=\color{ta3orange},       % comment style
  stringstyle=\color{ta3orange},         % string literal style
  emphstyle={[2]\color{ta2orange}},
  escapeinside=\$\$,            % if you want to add LaTeX within your code
  morekeywords={*,...}               % if you want to add more keywords to the set
  aboveskip=0pt,
  belowskip=0pt,
}}
\newcommand\biglistingstyle{\lstset{ %
    basicstyle=\small,
}}
\newcommand\altlistingstyle{\lstset{ %
    backgroundcolor=\color{black},
    basicstyle=\small\color{white},
    keywordstyle=\color{tachocolate},
    commentstyle=\color{tachameleon},
    stringstyle=\color{tachameleon},
    emphstyle={[2]\color{tabutter}},
    emphstyle={[3]\color{taorange}},
}}
\normallistingstyle
\newcommand\topshade{
    \begin{tikzpicture}[remember picture,overlay]
    \fill[black] (current page.west) rectangle +(100cm, -100cm);
    \end{tikzpicture}
    \vspace{-50pt}
    \biglistingstyle
}
\newcommand\halfshadenopause{
    \altlistingstyle\bigskip\bigskip\bigskip
}
\newcommand\halfshade{
    \pause\halfshadenopause
}
\newcommand\bottomshade{
    \normallistingstyle
}
\newcommand\sk{\par\bigskip\bigskip\par}
\newcommand\wh[1]{\only<#1>{\color{white}}}
\newcommand\tx[2]{\alt<#1>{\textcolor{ta3gray}}{\textcolor{ta3gray}}{\uncover<#1->{#2}}}
\newcommand\rd[2]{\alt<#1>{\textcolor{taskyblue}}{\textcolor{ta3gray}}{#2}}
\renewcommand\emph[1]{\textcolor{taskyblue}{#1}}

\begin{document}
\fontspec[Numbers=Lining]{Fertigo Pro}
\color{ta3gray}

\begin{center}
\title{eval()}
\author{Petr Viktorin}
\date{\today}

\frame{\color{ta3gray}
    \sk
    \textcolor{ta2gray}{\phantom{x}Python}
    \textcolor{taskyblue}{3.4}
    \textcolor{ta2gray}{\tiny \\[-0.25cm] (lightning talk)}
    \sk\sk
    \textcolor{ta2gray}{Petr Viktorin}\\[-0.25cm]
    \textcolor{ta2gray}{\tiny encukou@gmail.com}
    \sk
    \textcolor{ta2gray}{\tiny Pražské Pyvo, 2014-03-19}
}

\begin{frame}[fragile]
    Python 3.3

    \bigskip

    \emph{u}'string'
    \pause

    yield \emph{from}
    \pause

    venv
\end{frame}

\begin{frame}[fragile]
    Python 3.4

    \bigskip

    \tiny
    2014-03-16
\end{frame}

\begin{frame}[fragile]
    \emph{asyncio}
    \bigskip

    \pause

    "WSGI for event loops"

    \bigskip
    \pause

    twisted, tornado, gevent

    \bigskip
    \pause

    event loop, tasks, calbacks, futures/deferreds

    networking IO
\end{frame}

\begin{frame}[fragile]
    venv + \emph{ensurepip}
\end{frame}

\begin{frame}[fragile]
    \emph{enum}
    \bigskip

    \tiny
    \begin{verbatim}
    >>> from enum import Enum
    >>> class Color(Enum):
    ...     red = 1
    ...     green = 2
    ...     blue = 3

    >>> for c in Color:
    ...     print(c)
    Color.red
    Color.green
    Color.blue
    \end{verbatim}
\end{frame}

\begin{frame}[fragile]
    \emph{pathlib}
    \bigskip

    \tiny
    \begin{verbatim}
    >>> from pathlib import Path
    >>> p = Path('.')
    >>> p.isdir()
    True
    >>> p / 'slides.tex'
    PosixPath('./slides.tex')
    >>> (p / 'slides.tex').open().readline()
    '\\documentclass[20pt]{beamer}\n'
    \end{verbatim}
\end{frame}

\begin{frame}[fragile]
    \emph{statistics}
    \bigskip

    \tiny
    \begin{verbatim}
    >>> data = [1e30, 1, 3, -1e30]
    >>> sum(data) / len(data)
    0.0
    \end{verbatim}

    \begin{verbatim}
    >>> from statistics import mean
    >>> mean(data)
    1.0
    \end{verbatim}
\end{frame}

\begin{frame}[fragile]
    inspect \emph{CLI}
    \bigskip

    \tiny
    \begin{verbatim}
    $ python -m inspect re:search
    def search(pattern, string, flags=0):
        """Scan through string looking for a match
        to the pattern, returning a match object,
        or None if no match was found."""
        return _compile(pattern, flags).search(string)
    \end{verbatim}
\end{frame}

\begin{frame}[fragile]
    re.\emph{fullmatch}
    \bigskip

    \tiny
    \begin{verbatim}
    re.search('^abc$', string)
    re.match('abc$', string)
    re.fullmatch('abc', string)
    \end{verbatim}
\end{frame}

\begin{frame}[fragile]
    textwrap.\emph{shorten}
    \bigskip

    \tiny
    \begin{verbatim}
    >>> textwrap.shorten('Hello world! You are beautiful!', 20)
    'Hello world! [...]'
    \end{verbatim}
\end{frame}

\begin{frame}[fragile]
    \emph{CPython} improvements

    \bigskip
    \pause

    \verb+__del__+ + ref loops? \\
    no problemo

    \bigskip
    \pause

     30\% faster startup
\end{frame}

\begin{frame}[fragile]
    misc.

    \bigskip

    weakref.\emph{WeakMethod}

    \bigskip

    functools.\emph{partialmethod}

    \bigskip

    functools.\emph{singledispatch}

    \bigskip

    \emph{importlib} simplification

    \bigskip
    \pause

    with, signature
\end{frame}

\end{center}
\end{document}

